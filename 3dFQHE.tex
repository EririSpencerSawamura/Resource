\documentclass[aps,prl,twocolumn,amsmath,amssymb,showpacs,superscriptaddress,notitlepage,longbibliography]{revtex4-1}
\usepackage[colorlinks=true,linkcolor=blue,anchorcolor=red,citecolor=blue, urlcolor=blue]{hyperref}
\usepackage{bm}
\usepackage{graphicx}
\usepackage{color}
\usepackage{amsmath}
\usepackage{xy}

\begin{document}


\title{Decays of Majorana or Andreev oscillations induced by steplike spin-orbit coupling}


\author{Mingde Ren}
\affiliation{Shenzhen Institute for Quantum Science and Engineering and Department of Physics, Southern University of Science and Technology, Shenzhen 518055, China}

\author{Hai-Zhou Lu}
\email{Corresponding author: luhaizhou@gmail.com}
\affiliation{Shenzhen Institute for Quantum Science and Engineering and Department of Physics, Southern University of Science and Technology, Shenzhen 518055, China}
\affiliation{Peng Cheng Laboratory, Shenzhen 518055, China}
\affiliation{Shenzhen Key Laboratory of Quantum Science and Engineering, Shenzhen 518055, China}

\begin{abstract}
(This is Abstact)
\end{abstract}

\maketitle

\section{Introduction}
...

\section{Theory}
\subsection{Composite Fermion}
The electrons interacts strongly with each other in the fractional quantum Hall effect. However, Jain proposed a method to unify the fractional quantum Hall effect with the integer quantum Hall effect, i.e. the composite fermion approach. In this picture the quasi-particles are formed by electrons attached with an even number of vortices. The composite fermions can form integer quantum Hall effect or even their own fractional quantum Hall effect, which explains the original fractional quantum Hall effect formed by electrons.
\begin{equation}
\xymatrix{
    A'&&B'\\
    &A&&B\\
    C'&&D'\\
    &C&&D\\
    \ar"1,1";"2,2"%直接从(1,1)指向(2,2)
    \ar"1,3";"2,4"
    \ar"3,1";"4,2"
    \ar"3,3";"4,4"
    \ar"1,1";"1,3"
    \ar"2,2";"2,4"
    \ar"4,4";"4,2"
    \ar"3,1";"1,1"
    \ar"4,2";"2,2"
    \ar"2,4";"4,4"
    \ar"3,1";"3,3"|\hole%直接从(3,1)指向(3,3),带空洞
    \ar"3,3";"1,3"|\hole
}

\end{equation}



The idea of this method is to combine electrons with vortices:
\begin{equation}
\Psi(\vec{x}_1, \vec{x}_2, \cdots, \vec{x}_N) \mapsto \Psi(\vec{x}_1, \vec{x}_2, \cdots, \vec{x}_N)\exp(-2ip\sum_{i<j}\alpha_{ij}),
\end{equation}
where $i$,$j$ denotes the electrons, $\alpha_{ij}\doteq \arg(\vec{x}_i-\vec{x}_j)$ and $2p$ is the number of vortices attached.

Note that $\alpha_{ij}=\alpha_{ji}+\pi$ so that you still get a minus sign when exchanging two quasi-qarticles, i.e. they are still fermions.

This process shows the long range entanglement property of composite fermions: you cannot talk about a single composite fermion since its wave function depends explicitly on every other elections.

The phase factor can be understood as a gauge field: 
\begin{equation}
\vec{a}(\vec{x})\doteq \frac{\phi_0}{2\pi}\cdot 2p\sum_i \nabla \arg(\vec{x}-\vec{x}_i).
\end{equation}
Thus the effective vector potential that a composite fermion sees is $\vec{A}_{eff}=\vec{A}+\vec{a}$ and the effective magnetic field is $\vec{B}_{eff}\doteq\nabla\times\vec{A}_{eff}$.
The contribution of $\vec{a}$ is
\begin{equation}
\vec{b}=\nabla\times\vec{a}=2p\cdot\phi_0\sum_i\delta^2(\vec{x}-\vec{x}_i),
\end{equation}
so a vortex attached may also be interpreted as a flux quantum $\phi_0$.

Now assume that the electron density is uniform, the magnitude of effective magnetic field is then given by
\begin{equation}
B_{eff}=B-2p\rho\phi_0,
\end{equation}
where $\rho$ stands for electron density.

Then there will be Landau levels for composite fermoins and they may form their integer quantumm Hall effect. Let $\nu^*=\frac{\rho\phi_0}{\vert B_{eff} \vert}$ be the filling number of composite fermions $\nu=\frac{\rho\phi_0}{B}$ be that of electrons, then
\begin{equation}
\nu=\frac{\nu^*}{2p\nu^*\pm1},
\end{equation}
where -1 for the case that $B_{eff}$ is antiparallel to $B$.

The composite fermions may also form fractional quantum Hall effect so there is an iterative construction.

\subsection{3d IQHE}
In the previous work it is shown that 3d integer quantum Hall effect can be realised in anisotropic Weyl semimetals. The Hamiltonian is given by
\begin{equation}
H=D_1k_y^2+D_2(k_x^2+k_z^2)+A(k_x\sigma_x+k_y\sigma_y)+M(k_w^2-k^2)\sigma.
\end{equation}
After projecting onto the two Fermi arcs, one obtains the effective Hamiltonian:
\begin{equation}
\begin{split}
H_{arc}=&D_1k_w^2+vk_x+(D_2-D_1)(k_x^2+k_z^2),\\
H_{arc}'=&D_1k_w^2-vk_x+(D_2-D_1)(k_x^2+k_z^2),
\end{split}
\end{equation}
where $v$ is given by $v\doteq \frac{A\sqrt{M^2-D_1^2}}{M}$ and that $k_x, k_z$ satisfy the topological constrains:
\begin{equation}
\begin{cases}
k_x^2+k_z^2+2ak_x<k_w^2, \text{for }H_{arc},\\
k_x^2+k_z^2-2ak_x<k_w^2, \text{for }H_{arc}'.
\end{cases}
\end{equation}

The two Fermi arcs formed a loop so that the model form Landau levels. The physical picture is that the electrons are tunneling between the two surfaces and form integer quantum Hall effect.

The conductivity is given by
\begin{equation}
\sigma^{arc}_H=\frac{e^2}{h}sgn(R)sgn(eB)\lfloor\frac{S_I/(2\pi)^2}{eB/h}+\frac{1}{2}\rfloor,
\end{equation}
where $\lfloor\cdots\rfloor$ stands for rounding down, $R=D_2-D_1$ and $S_I$ stands for the area enveloped by the Fermi arcs in the momentum space.

\subsection{3d FQHE}
We now propose a new possibility to realise fractional quantum Hall effect in 3d.

The total number of electrons is given by $N=\frac{S_I}{(2\pi)^2}$ and thus the density is $\rho=\frac{S_I}{(2\pi L)^2}$. The filling number is $\nu=\frac{S_I/(2\pi)^2}{BL^2/\phi_0}=\frac{\rho\phi_0}{B}$.

Then the fractional quantum Hall effect should be observed when $\nu=\frac{\nu^*}{2p\nu^*\pm 1}$ for $p,\nu^*\in \mathbb{N}$. The corresponding magnetic field is $B=\frac{\nu^*}{2p\nu^*\pm1}\frac{1}{\phi_0\rho}$.

For example, $B=\frac{1}{3}\frac{1}{\phi_0\rho}, \frac{2}{5}\frac{1}{\phi_0\rho},\cdots$ 
\section{Conclusion}

%\bibliographystyle{apsrev4-1-etal-title}
%\bibliography{refs-Majorana}



%merlin.mbs apsrev4-1.bst 2010-07-25 4.21a (PWD, AO, DPC) hacked
%Control: key (0)
%Control: author (72) initials jnrlst
%Control: editor formatted (1) identically to author
%Control: production of article title (1) required
%Control: page (0) single
%Control: year (1) truncated
%Control: production of eprint (0) enabled
\begin{thebibliography}{73}%
\makeatletter
\providecommand \@ifxundefined [1]{%
 \@ifx{#1\undefined}
}%
\providecommand \@ifnum [1]{%
 \ifnum #1\expandafter \@firstoftwo
 \else \expandafter \@secondoftwo
 \fi
}%
\providecommand \@ifx [1]{%
 \ifx #1\expandafter \@firstoftwo
 \else \expandafter \@secondoftwo
 \fi
}%
\providecommand \natexlab [1]{#1}%
\providecommand \enquote  [1]{``#1''}%
\providecommand \bibnamefont  [1]{#1}%
\providecommand \bibfnamefont [1]{#1}%
\providecommand \citenamefont [1]{#1}%
\providecommand \href@noop [0]{\@secondoftwo}%
\providecommand \href [0]{\begingroup \@sanitize@url \@href}%
\providecommand \@href[1]{\@@startlink{#1}\@@href}%
\providecommand \@@href[1]{\endgroup#1\@@endlink}%
\providecommand \@sanitize@url [0]{\catcode `\\12\catcode `\$12\catcode
  `\&12\catcode `\#12\catcode `\^12\catcode `\_12\catcode `\%12\relax}%
\providecommand \@@startlink[1]{}%
\providecommand \@@endlink[0]{}%
\providecommand \url  [0]{\begingroup\@sanitize@url \@url }%
\providecommand \@url [1]{\endgroup\@href {#1}{\urlprefix }}%
\providecommand \urlprefix  [0]{URL }%
\providecommand \Eprint [0]{\href }%
\providecommand \doibase [0]{http://dx.doi.org/}%
\providecommand \selectlanguage [0]{\@gobble}%
\providecommand \bibinfo  [0]{\@secondoftwo}%
\providecommand \bibfield  [0]{\@secondoftwo}%
\providecommand \translation [1]{[#1]}%
\providecommand \BibitemOpen [0]{}%
\providecommand \bibitemStop [0]{}%
\providecommand \bibitemNoStop [0]{.\EOS\space}%
\providecommand \EOS [0]{\spacefactor3000\relax}%
\providecommand \BibitemShut  [1]{\csname bibitem#1\endcsname}%
\let\auto@bib@innerbib\@empty
%</preamble>
\end{thebibliography}%


\end{document}




















